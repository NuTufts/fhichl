\documentclass[12pt]{article}
\begin{document}
\section*{Purpose}
The purpose of the pretty print procedure is to produce
a human-readable parser-able text representation of a
parameter set object.
Its input is a parameter set object.
The output is the text representation.
An allowable implementation is permitted to accept
an object with file semantics as an additional input
and the text presentation is availed from the object
with file semantics after the procedure completes.
\section*{Details}
Various details not mentioned elsewhere:
\begin{enumerate}
\item {\textvisiblespace} represents a space (blank).
\item The only whitespace characters to be used outside the atoms are
      space (blank) and new-line.
\item The unit of indentation is two spaces.
\item Text in single quotes, not including the single quotes are literals.
      That is, `text' is the four characters `\verb+t+', `\verb+e+',
      `\verb+x+' and `\verb+t+', in that order.
\item Text in constant width font are literals.
      That is, {\tt text} is the four characters `\verb+t+', `\verb+e+',
      `\verb+x+' and `\verb+t+', in that order.
\item A digit is any one of the characters: `\verb+0+', `\verb+1+', `\verb+2+',
      `\verb+3+', `\verb+4+', `\verb+5+', `\verb+6+', `\verb+7+', `\verb+8+' or
      `\verb+9+'.
\item Stripping whitespace means reducing all sequences of whitespace not
      within a quoted string to a single space.
      Then, reduce `\verb+[+\textvisiblespace' to `\verb+[+',
      `\textvisiblespace\verb+]+' to `\verb+]+',
      `\verb+{+\textvisiblespace' to `\verb+{+' and
      `\textvisiblespace\verb+{+' to `\verb+{+'.
\end{enumerate}
\section*{Recognization of a numeric atom}
Definitions:
\begin{enumerate}
\item an exponent is either an `\verb+E+' or an `\verb+e+'
      followed by optionally either a `\verb|+|' or a `\verb|-|'
      followed by one to three digits
\item a number is any of these:
\begin{enumerate}
\item optionally either a `\verb|+|' or a `\verb|-|'
      followed by some digits
      followed by `\verb+.+'
      followed by optionally some digits
      followed by optionally an exponent.
\item optionally either a `\verb|+|' or a `\verb|-|'
      followed by optionally some digits
      followed by `\verb+.+'
      followed by some digits
      followed by optionally an exponent.
\item optionally either a `\verb|+|' or a `\verb|-|'
      followed by some digits
      followed by optionally an exponent.
\end{enumerate}
\item a complex number is a `\verb+(+' followed by a number 
      followed by a `\verb+,+' followed by a number followed by a `\verb+)+'
\end{enumerate}

If any of the following descriptions describes an atom completely,
then the atom is considered to be a numeric atom:
Starting at the beginning of the string, any of the following
\begin{enumerate}
\item {\tt nil}
\item {\tt true}
\item {\tt false}
\item optionally either a `\verb|+|' or a `\verb|-|' followed by {\tt infinity}
\item a number
\item a complex number
\end{enumerate}
and ending at the end of the string.
\section*{The external user visible procedure}
\begin{enumerate}
\item create a file semantics object, refered to herein as the destination
\item invoke the internal procedure with the following inputs:
\begin{enumerate}
\item the input object
\item the destination
\item the current indent with the value of a zero length string
\item the current name with the value of a zero length string
\item the current trailing mark with the value of a zero length string
\end{enumerate}
\item upon return from the internal procedure, if an error had occurred,
      then return an approriate error indication; otherwise, return the
      string value of the destination
\end{enumerate}
\section*{The internal procedure}
This description is worded as if each of the value type can pretty print itself.
This could be implemented by testing the value type and selecting the
appropriate formatting code.
If the value type is not recognized,
then an exception or error return must be created.

\begin{enumerate}
\item Append the current accumulated indent to the destination
\item If the name is not zero length,
      then append the name and `:\textvisiblespace' to the destination
\item Invoke the appropriate formatter according to the value type
\end{enumerate}

\subsection*{Atom}
If numeric, then append the atom text, the mark and a new-line to the
destination;
otherwise, append a `\verb+"+',
the atom text with `\verb+"+' replaced by `\verb+\"+'
and `\verb+\+' replaced by `\verb+\\+',
a `\verb+"+', the mark and a new-line to the destination
\subsection*{Sequence}
\begin{enumerate}
\item Create a new indent by concatenating the current indent with two spaces.
\item Create a file semantics object, refered to herein as subdestination
\item Append `\verb+[+' and a new-line to the subdestination
\item For each item in the sequence: invoke the internal procedure
      with the item, subdestination, new ident, a zero length string
      and if this is the last item of the sequence, then a zero length string;
      otherwise `\verb+,+'
\item Append the current ident, `\verb+]+' and the mark to the subdestination
\item Avail the new text from the subdestination
\item Strip the white space from the new text to form the stripped text.
\item If the stripped text is greater than 60 characters,
      then append the new text to the destination;
      otherwise append the stripped text to the destination
\item Append a new-line to the destination
\end{enumerate}
\subsection*{Parameter Set or Table}
\begin{enumerate}
\item Create a new indent by concatenating the current indent with two spaces.
\item Create a file semantics object, refered to herein as subdestination
\item Append `\verb+{+' and a new-line to the subdestination
\item For each key and value pair in the sequence: invoke the internal procedure
      with the value, subdestination, new ident, key
      and if this is the last item of the sequence, then a zero length string;
      otherwise `\verb+,+'
\item Append the current ident, `\verb+}+' and the mark to the subdestination
\item Avail the new text from the subdestination
\item Strip the white space from the new text to form the stripped text.
\item If the stripped text is greater than 60 characters,
      then append the new text to the destination;
      otherwise append the stripped text to the destination
\item Append a new-line to the destination
\end{enumerate}
\end{document}
