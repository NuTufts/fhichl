\documentclass{memarticle}
\usepackage{marvosym} % to define \MVAt
\usepackage{fancyvrb} % to import text files 

\newcommand{\docnumber}%
  {draft 3}

\newcommand{\doctitle}%
  {{\color{magenta}Specification of the Fermilab Hierarchical Configuration Language}}

\newcommand{\shorttitle}%
 {Specification of the FHiCL}

\newcommand{\authors}%
  {
  Ryan Putz
  }

\hypersetup%
  { pdfauthor={}%
  , pdftitle={\doctitle}%
  , pdfkeywords={}%
  , pdfsubject={}%
  }
\makeindex
\tightlist
\begin{document}
\topmatter
\newpage
\setlength{\parindent}{0in}
%       Body of document

\chapter{Introduction}
        \section{Purpose}
        {
                This document provides the formal specification
                for the \emph{Fermilab Hierarchical Configuration Language}, FHiCL \index{FHiCL}.
                This specification includes several aspects of FHiCL:
                \begin{itemize}
                        \item FHiCL Syntax
                        \item FHiCL Semantics
                        \item Canonical Value Representations
                \end{itemize}
                \par
                FHiCL is a customized language created for the storage of scientific parameter sets
                in a medium that can be easily understood and processed.
        }
        \section{Rationale}
        {
                FHiCL was developed in order to produce
                a standard configuration language for the storage,
                communication, 
                and manipulation
                of scientific parameter sets.
                \par
                The existence of a standard configuration language would allow for 
                the creation of programming language bindings that can read and process
                valid FHiCL documents, returning a parameter set to the user.
                %Why does this project exist?
        }
        
        \section{Scope of This Facility}
        {
                This project will include the development of a grammar specification for FHiCL
                (I.E. this document), creation of a basline parser (using Yacc and Bison), and the
                creation of various programming language bindings which shall read in
                FHiCL documents and create a parameter set.
                %What does this include? What does it not include?
        }
        \newpage
\chapter{FHiCL Syntax}{

        A baseline parser for FHiCL was constructed using Bison and Flex.
        Bison is a general purpose parser generator that converts a grammar description into
        a C program to parse that grammar.
        \par
        Flex is a lexical analyzer to process parsed tokens from the bison-generated C program.
        \par
        %Two commented lines are direct imports of file contents
        The FHiCL syntax is defined by the following Bison grammar:
        \begin{verbatim}
        \include{"bnf.y"}
        \end{verbatim}
        %\VerbatimInput[baselinestretch=1,fontsize=\footnotesize,numbers=left]{bnf.y}

        In this grammar,
        all uppercase names denote tokens.
        These tokens are defined by the following Flex specification:
        \begin{verbatim}
        \include{"bnf.l"}
        \end{verbatim}
        %\VerbatimInput[baselinestretch=1,fontsize=\footnotesize,numbers=left]{bnf.l}

        \section{Low-Level Entities}
                \textbf{ Note: } For all rules in this section,
                whitespace is not allowed between tokens.
		\index{number}
		\index{simple}
		\index{complex}
		\index{hexadecimal}
		\index{binary}
		\index{nil}
		\index{infinity}
		\subsection{Number}
			A \emph{number} in FHiCL is composed of several subtypes:
			\begin{itemize}
				\item Simple
				\item Complex
				\item Hexadecimal
				\item Binary
				\item Nil
				\item Infinity
			\end{itemize}
			\subsubsection{Simple}
				A \emph{simple} number is either an integer
				or a floating point number.
				\par
				EBNF:
				\begin{verbatim}
					float: [num]*[.][num]*
					integer: [1-9^.][0-9^.]*
					simple: float | integer
				\end{verbatim}
			\subsubsection{Complex}
				A \emph{complex} number is a tuple of simple numbers.
				\par
				EBNF:
				\begin{verbatim}
					complex: "(" simple "," simple ")" 
				\end{verbatim}
			\subsubsection{Hexadecimal}
				A \emph{hexadecimal} number is composed of two parts:
				\begin{enumerate}
					\item A prefix
					\item A numeric
				\end{enumerate}
				Where the numeric part is in base 16.
				\par
				The prefix for hexadecimal numbers is \emph{0x}
				or \emph{0X}.
				\par
				EBNF:
				\begin{verbatim}
					hex: (0x|0X)[0-9a-fA-F]+
				\end{verbatim}
			\subsubsection{Binary}
				A \emph{binary} number is similar to a hexadecimal number except that
				the numeric portion of the number is in base 2.
				Also, the prefix for \emph{binary} numbers is emph{0b} as opposed to
				\emph{0x} for hexadecimal numbers.
				\par
				EBNF:
				\begin{verbatim}
					bin: (0b|0B)[01]+
				\end{verbatim}
			\subsubsection{Nil}
				\emph{Nil} is FHiCL's implementation of \emph{null}.
				Valid forms for \emph{nil} are as follows:
				\begin{enumerate}
					\item nil
					\item Nil
					\item NIL
				\end{enumerate}
			\subsubsection{Infinity}
				\emph{Infinity}, in FHiCL, is simply a placeholder as the actual value
				will be filled in by whichever language binding is creating the parameter set.	
                \index{char}
                \subsection{Reserved and Special Characters}
                        A \emph{char} is one of:
                        \begin{enumerate}
                                \item any ASCII character except for:
                                \begin{itemize}
                                        \item double-quote ('') 
                                        \item reverse solidus (\textbackslash)
                                        \item control characters
                                \end{itemize}
				\index{printable}
                                \item (\emph{printable} characters)
                                \item one of a number escape sequences, noted below:
                                \begin{itemize}
                                        \item escaped double-quote (\textbackslash")
                                        \item reverse solidus (\textbackslash\textbackslash)
                                        \item solidus (\textbackslash/)
                                        %It says "noted below", but where?
                                \end{itemize}
                                There are a number of reserved char values:
                                \begin{itemize}
                                        \item colon ( : )
                                        \item double colon ( :: )
                                        \item left/right brace ( \{\} )
                                        \item left/right bracket ( [] )
                                        \item left/right paren ( () )
                                        \item at sign ( @ )
                                \end{itemize}           
                        \end{enumerate}
                \index{atom}
                \subsection{Atom}
%                       \bf Note: \rm The definitions of \emph{atom},
%                       \emph{table}, 
%                       and \emph{sequence} are mutually interdependent.
%                       \par
			\index{atom}
                        The most basic unit of FHiCL is the \emph{atom},
                        which is defined as: 
                \begin{verbatim}
        NIL: "nil" | nil | Nil | "Nil" | NIL | "NIL"
        BOOL_TOK: true | "true" | false | "false"
        REF: (@local:: | @db::) string
        atom: number | string | NIL | BOOL_TOK | REF
                \end{verbatim}
                        EBNF:
                \begin{verbatim}
        atom   =>    char | string
        string =>    alpha[alnum]* | digit[alnum]*
                \end{verbatim}
                        \par
                        \textbf{ Notes: }
                        \index{string}
                        \begin{itemize}
                                \item The canonical representation of an atom is a sequence of printable characters.
                                \item Every atom can be requested in canonical string form.
                                \item There are three valid syntaxes for a string in FHiCL:
                                \begin{enumerate}
                                        \item Alpha Start String - No quotes, string values must be \emph{simple}
                                                and contain no white space.
                                        \item Single-Quote - Surrounded by single quotes;
                                                all content is quoted verbatim.
                                        \item Double-Quote - Surrounded by double quotes;
                                                content may contain special escaped characters.
                                \end{enumerate}
                                \item The two special characters that are allowed in \emph{all} string forms are newline and tab.
                        \end{itemize}
                        
        \section{Mid-level Entities}
                \textbf{ Note: } For all rules in this section,
                whitespace is allowed only where specified by the whitespace token \emph{ws}.
                \index{comments}
                \subsection{Comments}
                        FHiCL comments are denoted by the \emph{\#} symbol,
                        or by \emph{\textbackslash\textbackslash}
                        which are placed at the beginning of the comment.
                        Comments may or may not be at the beginning on a line,
                        however, once comment notation is used, 
                        the rest of the line will be treated as a comment.
                        FHiCL comments are single-line,
                        and should be ignored by parsers.
                \index{names}
                \subsection{Names}
                        A \emph{name} is similar to a key in a key-value pair of a C++ mapping, or a Python dictionary.
                        In essence, a FHiCL name is an unquoted string that may contain only alphas and/or underscores.
                        \par\textbf{EBNF:}
                        \begin{verbatim}
        name: [a-zA-Z_]*
                        \end{verbatim}
                        \par\textbf{Example:}
                        \begin{verbatim}
        x: 1.0
                        \end{verbatim}
                        In this case, 
                        "x" is a \emph{name}.
                \index{Hierarchical Names}
                \index{hnames}
                \index{dot index}
                \index{bracket index}
                \subsection{Hierarchical Names}
                        A hierarchical name,
                        or \emph{hname} is a compound name 
                        using the \emph{dot index}
                        or \emph{bracket index}
                        to denote levels of scope.
                        \par
                        \emph{Dot index} is where a single period (".")
                        is used to denote access to a container's elements.
                        \par
                        \emph{Bracket index} is where a pair of brackets ( "\[ \]" )
                        are used to denote access to a sequence's elements.
                        Between the brackets is where an index must be given
                        as to which element in the sequence you wish to access.
                        \par
                        \textbf{Example:}
                        %is used in begin an \emph{override}:
                        \begin{verbatim}
        cont1:{x: 1.0 y: 2.0 z: 3.0}
        cont1.x : 5
        OR
        cont2:[1, 2, 3}
        cont2[0] : 1
                        \end{verbatim}
                        \par
                        EBNF:
                        \begin{verbatim}
        LBRACKET: "["
        RBRACKET: "]"
        BRACKET_INDEX: LBRACKET number RBRACKET
        DOT_INDEX: [.]
        hname => atom (DOT_INDEX|BRACKET_INDEX) atom
                        \end{verbatim}
                \index{value}
                \subsection{Value}
                        An element of type \emph{value} is either a single atom, 
                        a collection of atoms,
                        or a collection of associations.
                        Example:
                        \begin{verbatim}
        a : 1.0
        #Where "1.0" is the value of the atom named "a"
                        \end{verbatim}
                        
                        \par
                        EBNF:
                        \begin{verbatim}
        value => table|sequence|atom
                        \end{verbatim}  
                        \par
                        \textbf{ Note: } see definitions for \emph{table} 
                        and \emph{sequence} 
                        in the next section
        \section{High-Level Entities}
                \textbf{ Note: } For all the rules in this section,
                whitespace is allowed between any two tokens,
                and is not significant.
                \index{definition}               
                \subsection{Definition}
                        An element of type \emph{definition} is used to associate a value to a name.
                        The syntax of a \emph{definition} is:
                        \begin{verbatim}
        a : 1.0
                        \end{verbatim}
                        \par
                        EBNF:
                        \begin{verbatim}
        definition => (name|hname) COLON value
                        \end{verbatim}
                \index{table}  
                \subsection{Table}
                        Elements of type \emph{table} 
                        are space- or line-separated collections of definitions 
                        and are denoted by (possibly empty) braces:
                        \begin{verbatim}
        tab1:{a: 1.0 b: 2.0 c: 3.0}
                        \end{verbatim}
                        EBNF:
                        \begin{verbatim}
        table => LBRACE table_body RBRACE
        table_body => | table_items
        table_items => table_item | [table_item + "," + table_items]
        table_item => definition
                        \end{verbatim}
                        \par
                        \textbf{ Notes: }
                        \begin{itemize}
                                \item Tables may contain comments \textbf{ IF AND ONLY IF } 
                                                the table elements are line-separated. 
                                \item Comments cannot exist in between space-separated table elements.
                                \item two tables are the same when their hash code is the same 
                                                (the byte sequences fed into the hash must be identical).
                        \end{itemize}
                \index{sequence}
                \subsection{Sequence}
                        Elements of type \emph{sequence} 
                        are comma-separated collections of values 
                        and are denoted by (possibly empty) brackets:
                        \begin{verbatim}
         seq1:[a, b, c, d]
                        \end{verbatim}
                        EBNF:
                        \begin{verbatim}
         sequence => LBRACKET sequence_body RBRACKET
         sequence_body => | sequence_items
         sequence_items => sequence_item | [sequence_item + "," + sequence_items]
         sequence_item => value
                        \end{verbatim}
                        \par
                        \textbf{ NOTE: } Sequences \textbf{ CANNOT } contain comments.
                \newpage
		\index{FHiCL document|see{document}}
                \index{document}
		\index{.fcl}
		\index{table}
                \subsection{Document}
                        The \emph{document} is the highest-level construct 
                        in FHiCL.
                        Any implementation of a FHiCL parser
                        processes a \emph{document}
                        as if it were a single string.
                        \par
                        A file with the suffix ".fcl"
                        is considered to be a FHiCL document.
                        \par
                        A \emph{document} consists of exactly one,
                        possibly empty,
                        \emph{table} such as:
                        \begin{verbatim}
          #Document start
          main:{
                a: 1.0
                b: "hi"
                c: dog
               }
          #Document end
                        \end{verbatim}  
                        EBNF:
                        \begin{verbatim}
          document => table
                        \end{verbatim}
                        \par
                        Documents may have one or more prologs at the top of the document.
                        The only items that may occur before a prolog are comments and other prologs.
                \index{override}
		\index{table}
		\index{sequence}
                \subsection{Override}
                        An element of type \emph{override} is used to associate 
                        an existing element with a new value,
                        or to create a new element in a \emph{table} or \emph{sequence}.
                        The syntax for an override:
                        \begin{verbatim}
          a: 1.0 #Declaration and initialization
          a : 5.0 #Override (Assignment)
                                
          OR
                                
          tab1:{ a:1 b:2 c:3 }
          tab1.d : 5 #Creating a new element 'd' in table 'tab1'
                                
          OR
                                
          seq1:[ 1, 2, 3 ]
          seq1[3] : 5 #Creating a new element '5' in sequence 'seq1'
                        \end{verbatim}
                        \par
                        EBNF:
                        \begin{verbatim}
          override => (name|hname|DOTINDEX|BRACKETINDEX) COLON value
                        \end{verbatim}
			\index{name}
			\index{hname}  
                        \textbf{ Note: } the \emph{name} for an override 
                        is an \emph{hname}.
                \newpage
                \index{include}
		\index{FHiCL}
		\index{document}
                \subsection{Include}
                        In order to import values from external documents into a FHiCL document,
                        an \emph{include} statement is used to tell which file's values should be inserted into the document.
                        \par
                        A FHiCL \#include statement differs from the C++ \#include statement
                        in that the FHiCL \#include acts
                        more as a union of two documents
                        , as opposed to just allowing one file to access another.
                        \par
                        The \emph{include} statement syntax is as follows:
                        \textbf{Example:}
                        \begin{verbatim}
           //This is a valid include statement:
           #include "filename.ext"
           #include "../test1.fcl"
           #include "tests/pass/test2.fcl"
                     
           //These are invalid include statements:
           #include filename.ext
           //include "filename.ext"
           #include"filename.ext"
           include "filename.ext"
           #includefilename.ext
           
                        \end{verbatim}
                        \par
                        Where the quoted string "filename.ext" represents the file name and file extension of the included file.
                        \par
                        \subsubsection{Default Directory}
                                The default directory from which all searches for included files are performed
                                is the same directory from which the FHiCL parser is run.
                                \par
                                Therefore, suppose we have the following directory structure:
                                \begin{verbatim}
            /fhicl
            |---parser
            |---/testFiles
                |----test1.fcl
                |----test2.fcl
                                \end{verbatim}
                                \par
				\index{parser}
                                Assuming that \emph{parser} is the parsing program for FHiCL
                                and that test1.fcl, and test2.fcl are FHiCL documents.
                                If the file \emph{test1.fcl} has the following contents:
                                \begin{verbatim}
            a:1
            b:2
            c:3
                                \end{verbatim}
                                And the file \emph{test2.fcl} having the following contents:
                                \begin{verbatim}
            a:0
            d:4
                                \end{verbatim}
				\index{include}
                                In order to \emph{include} the file \emph{test2.fcl} in emph{test1.fcl},
                                The \#include statement would look like this:
                                \begin{verbatim}
            #include "testFiles/test2.fcl"
                                \end{verbatim}
				\index{parser}
                                Since the default directory from which the search for \emph{test2.fcl} is begun
                                is \emph{/fhicl}, which is where the program \emph{parser} is located.
                                \par
                                Now if the directory structure looked like this:
                                \begin{verbatim}
            /fhicl
            |---parser
            |---test1.fcl
            |---test2.fcl
                                \end{verbatim}
                                Then the \#include statement would look like this:
                                \begin{verbatim}
            #include "test2.fcl"
                                \end{verbatim}
                        \textbf{ NOTES: }
                        \begin{itemize}
                           \item There is exactly one space between '\#include' and 'filename.ext'.
                           \item Also, the filname must be enclosed in double quotes.
                           \item Any deviation from the include statement syntax will result in a parse failure.
                           \item Circular or repetive includes are \emph{not} supported
                                 and should be checked for by the parser.
                           \item Included values can be overridden
                                 and can override values that are within the same scope
                                 and share the same name.
                           \item Includes must be on their own line, otherwise they will be treated as comments
                        \end{itemize}
                \index{prolog}
		\index{BEGIN\_PROLOG}
		\index{END\_PROLOG}
                \subsection{Prolog}
                        A \emph{Prolog} is a construct which exist at the start of a FHiCL document.
                        A Prolog's boundaries are denoted by the use of \emph{BEGIN\_PROLOG} and \emph{END\_PROLOG}.
                        All data within a Prolog may not be modified outside of the Prolog.
                        \par
                        Below is an example of a valid FHiCL Prolog:
                        \begin{verbatim}
             BEGIN_PROLOG
             x :5
             y :6
             END_PROLOG
                        \end{verbatim}
                \index{reference}
                \subsubsection{Reference}
                        In order to associate a name
                        with the value of a pre-existing definition
                        the use of the FHiCL \emph{reference} notation is required:
                        \begin{verbatim}
             @local::
             OR
             @db::
                        \end{verbatim}
                        Example:
                        \begin{verbatim}
             x : 5
             y : @local::x
             z : @db::x
                        \end{verbatim}
                        \par
                        References point to the most-recently encountered variable
                        with a matching name.
                        Reference names must be extremely specific
                        in which value they are pointing to.
                        \par
                        For example, if we have a table \emph{tab1} such as:
                        \begin{verbatim}
             tab1:{ a:1 b:2 c:3 }
                        \end{verbatim}
                        and we want to set an outside variable to the value of \emph{a} in \emph{tab1}.
                        The reference for this would look like:
                        \begin{verbatim}
             tab1:{ a:1 b:2 c:3 }
             x : @local::tab1.a
                        \end{verbatim}
                        And this would give us a resulting parameter set of \emph{x : 1}
                        \par
                        In situations where an element in a prolog shares a name with an element in the document body,
                        any references made to a variable of the same name will result in a reference look-up to the element
                        in the document body.
}
\chapter{FHiCL Semantics}{
        \section{High-level Result of a Successful Parse}
                \index{document}
                \index{table}
                \index{definition}
                \index{override}
                
                The result of parsing a \emph{document}
                is a single \emph{table}.
                The \emph{definition}s and \emph{override}s
                appearing before the top-level \emph{table}
                are intended to allow the user
                to supply values to be substituted into elements in the \emph{table}.
                The \emph{definition}s and \emph{override}s
                appearing after the top-level \emph{table}
                are intended to allow the user
                to replace values in that table.

        \section{Representation of Atoms}
                \index{atom}
                \index{nil}
                \index{reference}
                In the parse results,
                all \emph{atom}s
                except for \texttt{nil} and \emph{reference}
                are represented
                as character strings.
                The atom \texttt{nil} is represented by a 
                value specified by the binding for a given programming language.
                The resolution of \emph{reference}s is described in section \emph{Resolution of References} below.
                \par
                Each language binding
                provides its own mechanism
                for turning atoms of type \emph{integer}, \emph{real} and \emph{complex}
                from their string representation
                into the appropriate numerical representation.
	\section{Canonical Forms}
		\index{boolean, canonical}
		\index{true, canonical}
		\index{false, canonical}
		\index{integer, canonical}
		\index{floating point, canonical}
		\index{hex, canonical}
		\index{bin, canonical}
		\index{string, canonical}
		\index{canonical form}
		\subsection{Canonical Booleans}
			Boolean values, whether entered with quotes or not,
			will be stored in the following form (EBNF):
			\begin{verbatim}
				bareT: true
				quoteT: "true"
				bareF: false
				quoteF: "false"
				bool: (["]bareT | bareF["]) | (quoteT | quoteF)
			\end{verbatim}
		\subsection{Canonical Numbers}
			Numeric values in FHiCL each have their own canonical form based on their type.
			\subsubsection{Integer}
				Integers in FHiCL \emph{may} have leading zeros,
				however the canonical form will strip any leading zeros from the integer.
				Also, if an integer is outside of the \emph{small range} (which is 1,000,000),
				its canonical form will be a floating point number using scientific notation.

			\subsubsection{Floating Point}
				Floating point numbers in FHiCL may also have leading zeros,
				with the canonical form stripping out all but one leading zero.
				Also, if a floating point number can be represented as an integer
				and is within the \emph{small range}, 
				then the floating point number will be canonically stored as an integer.

			\subsubsection{Hexadecimal and Binary}
				The numeric portions of \emph{hexadecimal} and \emph{binary} numbers will have the leading zeros stripped.
				The prefix used to identify both types of numbers \emph{do not} count as a leading zero.
				The canonical form of hexadecimal and binary numbers in FHiCL is as follows:
				\par
				EBNF:
				\begin{verbatim}
					hex: (0x|0X)[1-9a-fA-F][0-9a-fA-F]*
					bin: (0b|0B)[1][10]*
				\end{verbatim}
			\subsubsection{General Notes}
				Negative(-) signage is supported in FHiCL and is kept in the canonical form.
				However, positive(+) signage, while it is supported, is stripped when in canonical form. 
				So, \emph{$-$infinity} in canonical form is still \emph{$-$infinity}.
				However, emph{$+$infinity} becomes just \emph{infinity} in canonical form.
		\subsection{Canonical Strings}
			Canonical form for all \emph{string}s is a string representation of the characters.
			\par
			Notes:
			\begin{itemize}
				\item Stirng concatenation operatiosn are permitted, but only for quoted string values.
				\item No unquoted white space is permitted.
				\item Quotes for string values can be omitted if the string value is considered to be 'simple'.
				\item A 'simple' string is made up of only underscores and alphabetic characters.
			\end{itemize}
        \index{reference}
        \index{hname}
        \index{local}
        \index{db}
        \index{document}
        \index{value}
        \section{Resolution of \emph{References}s}
                Atoms of type \emph{reference} are replaced
                by the value indicated by the \emph{hname} part of the \emph{reference},
                where the environment in which the \emph{hname} is evaluated is determined
                by the \texttt{db} or \texttt{local} at the end of the \emph{reference}.
                \par
                The presence of \texttt{local} indicates 
                that the scope in which the \emph{hname} is to be evaluated
                is the previously-read \emph{document} text.
                The presence of 
                \texttt{db} indicates
                that the scope in which the \emph{hname} is evaluated
                is the single database
                to which the parser has access.
                \par
                If the parser has no access to a database,
                and a \emph{reference} which ends in \texttt{db} is encountered,
                a parse failure results.
                If,
                in the appropriate scope,
                the \emph{hname} in a \emph{reference} does not resolve to any \emph{value},
                a parse failure results.
        \section{Issues with Leading Zeros and Canonical Representation}
                As a rule, leading zeros are not allowed
                in any situation where a number may be misinterpreted as a non-base-10 number
                with the inclusion of (a) leading zero(s).
                \par
                This rule only applies to numbers that may be represented as a base-10 integer.
                Floating point, binary, hexidecimal, and octal numbers may have leading zeros.
                Exponential numbers may have leading zeros, but if they are representable as a base-10 integer,
                their canonical form will be in integer form.
                \par
                The rationale for this rule is that in some programming languages,
                a leading zero is used to denote a non-base-10 number,
                I.E. "0x" is used to denote a hexidecimal number.
}   
\chapter{Features of Programming Language Bindings}
        \section{Processing}
                Each programming language binding for FHiCL must be able to produce a parameter set
                in the standard FHiCL syntax. 
        
        \section{Output}
                Each language binding shall return a native container construct closest to that of the FHiCL table.
                The returned container shall contain a valid FHiCL parameter set.

        \section{Storage}    
                Storage of parsed results from each program language binding shall be in th standard FHiCL syntax as defined above.
                FHiCL documents are to be stored in files with the suffix ".fcl".

\chapter{General Requirements}
        \section{Additional Requirements for Dynamically Typed Languages}
                Tables and sequences should be represented by a built-in type of the
                programming language.

                If the target programming language has a standard JSON library, we
                want to make sure that our constructs can be translated to JSON format
                and back without use of any FHiCL-specific library.

                It is important that code that uses the representation of a
                \emph{table} not need any FHiCL-specific code.

\chapter{Output Requirements}
        \section{Output Intended for Human Reading}
                "Pretty Printers" must make use of newlines and indentation throughout parameter set output.
                The use of newlines and indentation between table elements, individual associations, include statements
                and comments is required.
        \section{Output Intended for Machine Reading}
                Output for use by machine(s) is to be machine parsable,
                have an ASCII dump facility and platform neutral.
                Machine output is to be exclude unnecessary elements such as comments.

\chapter{Glossary}
                \section{Alphas}
                        An \emph{alpha} is any of the ASCII characters a-z or A-Z.
                \section{Digits}
                        A \emph{digit} is any of the ASCII characters 0-9.
                \section{White Space}
                        A \emph{ws} is one of the three whitespace characters: space/tab, newline, and line return. 
                \section{Alphanumerics}
                        An \emph{alnum} is any of the ASCII characters a-z, A-Z, 0-9 or other \emph{printable} characters
\printindex
\end{document}
