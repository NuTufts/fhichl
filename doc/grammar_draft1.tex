\documentclass{memarticle}
\usepackage{marvosym} % to define \MVAt
\usepackage{rail}
\newcommand{\docnumber}%
  {draft 1}

\newcommand{\doctitle}%
  {DRAFT---{\color{magenta}Specification of the Fermilab Configuration Language}---DRAFT}

\newcommand{\shorttitle}%
 {DRAFT---Specification of the FCL---DRAFT}

\newcommand{\authors}%
  {
  Marc~Paterno
  }

\hypersetup%
  { pdfauthor={}%
  , pdftitle={\doctitle}%
  , pdfkeywords={}%
  , pdfsubject={}%
  }

\tightlists

%
% Stuff to control the behavior of rail.
%
\railoptions{-hi}   % \railoptions{-h}
\railparam{\setlength{\leftmargin}{\leftmargini}}
\railnamefont{\rmfamily\itshape\color{magenta}}
%
\railalias{alphac}{a-zA-Z}
\railalias{at}{\MVAt}
\railalias{cbrace}{\}}
\railalias{cquote}{'}
\railalias{cr}{\char"5C\char"5C}
\railalias{digitc}{0-9}
\railalias{dollar}{\$}
\railalias{dquote}{"}
\railalias{greater}{\textgreater}
\railalias{hat}{\textasciicircum}
\railalias{less}{\textless}
%\railalias{nonzero}{1-9}
\railalias{obrace}{\{}
\railalias{oquote}{`}
\railalias{printc}{printable}
\railalias{rsolidus}{\textbackslash}
\railalias{space}{\textvisiblespace}
\railalias{star}{\textasteriskcentered}
\railalias{tilde}{$\sim$}
\railalias{uline}{\_}
\railalias{vbar}{\textbar}
%
\railterm{alphac,at,cbrace,cquote,cr,dollar,dquote,greater,hat,less,obrace,oquote,rsolidus,space,star,tilde,uline,vbar}

\makeindex

\begin{document}
\topmatter
\chapter{Introduction}

\section{Purpose of this document}
This document provides the formal specification
for the \emph{Fermilab Configuration Language},
FCL.
It uses extended Backus-Naur formt (EBNF) to describe the language grammar.

\section{Notation used in this document}

Any \emph{nonterminal} is formatted in a square box, like:
\begin{rail}
  nonterminal;
\end{rail}

Any \emph{terminal} text is formatted in a box with rounded corners, like:
\begin{rail}
  'terminal';
\end{rail}

%A list of all terminals of FCL is provided in Appendix~\ref{app:terminals}

Arrows are used to show the direction in which diagrams should be read.

\chapter{Configuration language syntax}

\section{High-level entities}

For all the rules in this section,
whitespace is allowed between any two tokens,
and is not significant.

The \emph{document} is the highest-level construct
in FCL.
Any implementation of an FCL parser
processes a \emph{document}
as if were a single string.

A \emph{document} consists of a mixture of
zero or more \emph{definitions}
and \emph{overrides},
and exactly one \emph{table},
which are all comma-separated:
\begin{rail}
document: ((definition ',' |override ',' ) * )  table \\ (( ',' definition| ',' override) * ) 'EOF';
\end{rail}

An element of type \emph{definition} is used to
create a new element:
\begin{rail}
	definition: name ':' value;
\end{rail}

An element of type \emph{override} is used to
change the value of an existing element,
or to create a new element in a \emph{table} or \emph{sequence}.
\begin{rail}
  override: hname ':' value;
\end{rail}

An element of type \emph{value} is either a \emph{table},
a \emph{sequence},
or an \emph{atom}:
\begin{rail}
value: table | sequence | atom;
\end{rail}
The definitions of \emph{table}, \emph{sequence} and \emph{atom} are mutually interdependent.

Elements of type \emph{table} are denoted by (possibly empty) braces;
the elements within the braces
are \emph{definitions}
which are separated by commas:
\begin{rail}
table: obrace (definition * ',') cbrace;
\end{rail}

Elements of type \emph{sequence} are denoted by (possibly empty) brackets;
the elements within the brackets can be any value,
and are separated by commas:
\begin{rail}
sequence: '[' (value * ',') ']';
\end{rail}

\section{Mid-level entities}

For all rules in this section,
whitespace is allowed only where specified by the whitespace token \emph{ws}.

A \emph{name} is used to begin a \emph{definition}:
\begin{rail}
name: (alpha | uline) ((alpha|digit| uline) *);
\end{rail}

\break
A hierarchical name,
or \emph{hname},
is used in begin an \emph{override}:
\begin{rail}
  hname: name (('.' name|'[' integer ']')+);
\end{rail}

The most basic unit of the configuration language is the \emph{atom}:
\begin{rail}
atom: ref | string | integer | real | complex | 'true' | 'false' | 'null';
\end{rail}


\begin{rail}
  ref: (name|hname) at ('db'|'local');
\end{rail}

\begin{rail}
string: dquote (char +) dquote;
\end{rail}

\begin{rail}
integer: '-'? ((nonzero (digit *))|('0'));
\end{rail}
Note that a leading $+$ is not legal,
nor is a leading 0 legal unless it is the sole character.

\begin{rail}
%real: integer ((('.' (digit +))*)|((('.' (digit +))*) (('e'|'E') ('+'|'-')? (digit+))?);
 real: integer ( mantissa | exponent | (mantissa exponent) );
\end{rail}

\begin{rail}
	mantissa: '.' (digit +);
\end{rail}

\begin{rail}
	exponent: ('e'|'E') ('+'|'-')? (digit+);
\end{rail}

\begin{rail}
complex: '(' ('ws'*) (real|integer) ('ws'*) ',' \\ ('ws'*) (real|integer) ('ws'*) ')';
\end{rail}

\section{Low-level entities}

For all rules in this section,
whitespace is not allowed between tokens.

A \emph{char} is one of:
\begin{enumerate}
\item any ASCII character except for
the double-quote ('') 
or the reverse solidus (\textbackslash)
or control characters
(\emph{printable} characters), or
\item an escaped double-quote or reverse solidus or solidus
(escaping is done with a reverse solidus), or
\item one of a number escape sequences, noted below.
\end{enumerate}
\begin{rail}
char: printable|(rsolidus (dquote|'/'|'b'|'f'|'n'|'r'|'t'));
\end{rail}

\begin{rail}
	printable: alpha | digit | punctuation | symbol;
\end{rail}

An \emph{alpha} is any of the ASCII characters \texttt{a}--\texttt{z}
or \texttt{A}--\texttt{Z}.
\begin{rail}
alpha: alphac
\end{rail}

A \emph{digit} is any of the ASCII characters \texttt{0}--\texttt{9}.
We distinguish between zero digits and nonzero digits:
\begin{rail}
digit: '0' | nonzero;
\end{rail}
\begin{rail}
	nonzero: '1-9';	
\end{rail}
\begin{rail}
	punctuation: '!' | oquote | cquote | '(' | ')' | ',' | '.' | ':' | ';' | '?' | '[' | ']' | obrace | cbrace;
\end{rail}

\begin{rail}
  symbols: 'space' | '\#' | dollar | '\%' | '\&' | star | '+' | '-' | '/' | less | '=' | greater | vbar | at | rsolidus | hat | uline | tilde;
\end{rail}

A \emph{ws} is one of the three whitespace characters:
\begin{rail}
	ws: 'space' | 'tab' | 'newline';
\end{rail}

\chapter{Configuration language semantics}
\section{High-level result of a successful parse}

The result of parsing a \emph{document}
is a single \emph{table}.
The \emph{definition}s and \emph{override}s
appearing before the top-level \emph{table}
are intended to allow the user
to supply values to be substituted into element in the \emph{table}.
The \emph{definition}s and \emph{override}s
appearing after the top-level \emph{table}
are intended to allow the user
to replace values in that table.

\section{Representation of atoms}
In the parse results,
all \emph{atom}s
except for \texttt{null} and \emph{ref}
are represented
as character strings.
The atom \texttt{null} is represented by a 
value specified by the binding for a given programming language.
The resolution of \emph{ref}s is described in section~\ref{sec:refs} below.

Each language binding
provides its own mechanism
for turning atoms of type \emph{integer}, \emph{real} and \emph{complex}
from their string representation
into the appropriate numerical representation.

\section{Resolution of \emph{ref}s\label{sec:refs}}
Atoms of type \emph{ref} are replaced
by the value indicated by the \emph{hname} part of the \emph{ref},
where the environment in which the \emph{hname} is evaluated is determined
by the \texttt{db} or \texttt{local} at the end of the \emph{ref}.

The presence of \texttt{local} indicates 
that the scope in which the \emph{hname} is to be sought
is the previously-read \emph{document} text.
The presence of \texttt{db} indicates
that the scope in which the \emph{hname} is evaluated
is the single database
to which the parser has access.
If the parser has no access to a database,
and a \emph{ref} which ends in \texttt{db} is encountered,
a parse failure results.
If,
in the appropriate scope,
the \emph{hname} in a \emph{ref} does not resolve to any \emph{value},
a parse failure results.

\appendix

\chapter{Differences between JSON and FCL\label{app:differences}}
The language described is,
by intention,
similar but not identical to the Javascript Object Notation (JSON),
described at \url{http://www.json.org}.

Where JSON uses the name \emph{object},
we use the name \emph{table}.
Where JSON uses the name \emph{array}
we use the name \emph{sequence}.

The configuration language differs from JSON in several ways.
We draw special attention to the following.
\begin{enumerate}

\item JSON requires that the names of members in an object be strings,
which in JSON are always delimited by double quotes.
In the configuration language,
names of members of objects are not quoted,
and are subject to different constraints;
approximately stated,
names must be suitable as variable names
in commonly-used programming languages.
See the grammar specification below
for an exact description of the constraints.

\item JSON allows documents to contain any Unicode character.
The configuration language restricts documents 
to contain only printable ASCII characters.
This choice was made for the configuration language
because some of the languages for which we require bindings
do not have convenient support for Unicode.

\item JSON does not directly support complex numbers.
The configuration language has direct support 
for specification of values
that are complex numbers.

\item JSON recognizes \emph{number} as a primitive data type.
In the configuration language,
numbers and strings are united into a common type \emph{atom}.
This choice was made for the configuration language
because we need to support producing a printed representation
for every value
that is identical to the representation in the configuration document.
\end{enumerate}

%\chapter{Terminals\label{app:terminals}}
%\begin{fixme}
%	Would it be useful to have a table of all FCL terminals here?
%	If so, what information should be in the table?	
%\end{fixme}

\printindex

\end{document}
