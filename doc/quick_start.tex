%\documentclass{memarticle}
\documentclass{article}
\usepackage{marvosym} % to define \MVAt
\newcommand{\docnumber}%
  {draft 1}

\newcommand{\doctitle}%
  {DRAFT---{\color{magenta}Quick Start Guide for the Fermilab Hierarchical Configuration Language}---DRAFT}

\newcommand{\shorttitle}%
 {DRAFT---FHiCL Quick Start Guide---DRAFT}

\newcommand{\authors}%
  {
  Ryan Putz
  }

%\hypersetup%
%  { pdfauthor={}%
%  , pdftitle={\doctitle}%
%  , pdfkeywords={}%
%  , pdfsubject={}%
%  }

%\tightlists

\makeindex

\begin{document}
%\topmatter
\setlength{\parindent}{0in}
%       Title stuff
\title{Quick Start Guide}
\author{Ryan Putz}
\maketitle
\newpage
%       Body of document

\tableofcontents
\newpage

\section{Introduction}
	The purpose of this document is to train users in the proper useage
	of the Fermilab Hierarchical Configuration Language (FHiCL).
	This Quick Start Guide will explain and demonstrate the major syntax and semantics
	of the language so that users may create valid FHiCL documents with ease.
\section{Comments}
	FHiCL comment notation bears similarity to that of PHP
	with the exception of FHiCL not supporting the C/C++ block comment notation.
	FHiCL comments have two notations:
	\begin{itemize}
		\item The pound/hash sign ( \# )
		\item Two backslashes ( \textbackslash\textbackslash )
	\end{itemize}
	\par
	The former notation is the same notation used in such programming and scripting languages as
	Perl, Python, and Ruby.
	\par
	The latter notation is the C++ inline comment notation.
	\par
\section{Associations}
        \subsection{Names}
		FHiCL names are simple, non-quoted strings.
		Names may not begin with a digit,
		but may begin with an underscore.
		\par
		Illustrated in Example 1.1.0,
		you will find a list of examples of valid FHiCL name values.
		\par
		Example 1.1.0:
		\begin{verbatim}
			a
			name
			the_dog
			_a_cat
			a_fish_
			_a_bird_
		\end{verbatim}
		\par
		There is also an advanced type of FHiCL name called 
		\emph{hierarchical name} or \emph{hname} for short.
		These hnames are compound names linked together through the use of
		a \emph{dot index} or a \emph{bracket index}.
		Hnames are used to access members of FHiCL containers
		and their associated values.
		\par
		Example 1.1.1:
		\begin{verbatim}
			name.first = "beth"
			name.last = "johnson"
			scores[0] = 5
			scores[3] = 6
		\end{verbatim}
		\par
		We will see more in-depth examples of hname useage later on.
	\subsection{Values}
		In contrast, FHiCL values are extremely diverse.
		FHiCL values can be any one of the following:
		\begin{itemize}
			\item Single character (quoted or unquoted)
			\item String (quoted or unquoted)
			\item Number
			\item Table
			\item Sequence
			\item Document
			\item 
		\end{itemize}
		\par
		In Example 1.2.0, there are a few examples of valid FHiCL values.
		Example 1.2.0:
		\begin{verbatim}
			a
			ab
			string
			"string"
			'string'
			12345
			"123abc"
			'123abc'
		\end{verbatim}
	\subsection{The Syntax}
		In most programming languages there is the idea of an \emph{assignment}. 
		In FHiCL, however, instead of assigning a value to a name, 
		we are \emph{associating} a value with a name; 
		much like the relationship between a key and a value in a C++ std::map.
		\par
		The basic association is the backbone of the FHiCL syntax.
		A basic FHiCL association has three parts:
		\begin{enumerate}
			\item A name followed by...
			\item The colon symbol ( : ); followed by...
			\item A value.
		\end{enumerate}
		\par
		Or in EBNF it would look like this:
		\begin{verbatim}
			association => name COLON value
		\end{verbatim}
		\par
		There are slight variations on this model used throughout the language,
		but the basic premise does not change. 
		We will see examples of these variations later on.
	\subsection{Examples}
		Let's say that we want to declare three new FHiCL elements: a, b, c
		with values of 1, 2, 3, respectively. 
		Here is how that would look in a valid FHiCL document:
		\par
		Example 1.4.0:
		\begin{verbatim}
			a : 1
			b : 2
			c : 3
		\end{verbatim}
		\par
		Now we have three new FHiCL name-value pairs!
		Please note that in \emph{most} aspects of FHiCL,
		the use of whitespace is optional.
		\par
		So this would also be a valid way to write the associations in Example 1.0:
		\par
		Example 1.4.1:
		\begin{verbatim}
			a:1
			b:2
			c:3
		\end{verbatim} 
 		\par
		Here are a few more examples of valid FHiCL associations using various types of strings:
		\par
		Example 1.4.2:
		\begin{verbatim}
			dog : "Wish Bone"
			cat : 'QT McWhiskers'
			fish : Klaus
		\end{verbatim}
		Please note that unquoted string values must be \emph{simple} in that they may not contain
		any white space.
		Double-quoted strings may contain special escaped characters,
		and single-quoted strings are quoted verbatim.
\section{Overrides}
	In FHiCL, the term \emph{override} has special meaning and its use is two-fold:
	\begin{enumerate}
		\item Reassociation of an existing name to a new value.
		\item Creation of a new element within a table or sequence.
	\end{enumerate}
	\par
	The syntax for a FHiCL override statement is identical to that of a FHiCL association.
	The only difference between the two is that in an override statement,
	the \emph{name} portion of the statement must be the name of a pre-existing element
	which exists at the same level of scope as the override statement.
	\par
	Please note that the use of an hname allows an override statement to descend through layers of scope,
	but there is no way to ascend through layers of scope.
	\par
	Example 2.0.0:
	\begin{verbatim}
		x : 5
		x : 6
	\end{verbatim}
\section{Tables And Sequences}
	Two major constructs in FHiCL are the \emph{table} and the \emph{sequence}.
	Each of these containters has a unique syntax for declaration and access.

	\subsection{Tables}
		A FHiCL table is a container for name-value pairs
		and is denoted by (possibly empty) braces.
		\par
		Example 3.1.0
		\begin{verbatim}
			{
			   a : 5
			   b : 6
			   c : 7
			}
		\end{verbatim}
		\par
		Elements of a FHiCL table may \emph{not} be comma-separated.
		The elements may be separated by either whitespace or newline.
		Therefore it would also be valid to write \emph{table1} like this:
		\par
		Example 3.1.1
		\begin{verbatim}
			{ a : 5 b : 6 c : 7 }
			OR
			{ a:5 b:6 c:7 }
			NOT
			{ a:5, b:6, c:7}
		\end{verbatim}
		\par
		FHiCL tables also have a unique quality in that they may contain comments.
		This is true only if the elements of the table are line-separated.
		Example 3.1.2
		\begin{verbatim}
			{
			   a : 5
			   #This is a comment
			   b : 6
			   //This is a comment
			}

			NOT

			{ a:5 #this is a comment b:6}
		\end{verbatim}
		\par
		The second case is invalid since FHiCL comments take the rest of the line,
		the closing brace for the table will be interpreted as part of the comment.
		\par
		In order to acces members of a FHiCL table, use of an \emph{hname}
		with the \emph{dot index} is needed.
		\par
		Useage of the \emph{dot index} notation requires that the table be associated with a name.
		\par
		For instance, if we wanted to access \emph{a} from \emph{table1} in Example 3.1.0,
		and override the value of \emph{a} to 4, it would look like this:
		\par
		Example 3.1.3:
		\begin{verbatim}
			table1:{
                           a : 5
                           b : 6
                           c : 7
                        }
			table1.a : 4
		\end{verbatim}
		\par
		Similarly, if we wanted to add or \emph{override} in a fourth member to \emph{table1}
		(we'll call it \emph{d} and we'll give it a value of 8)
		it would look like this:
		\par
		Example 3.1.4:
		\begin{verbatim}
			table1.d : 8
		\end{verbatim}
		\par
	\subsection{Sequences}
		A FHiCL sequence is a container of values
		and is denoted by (possibly empty) brackets.
		\par
		Example 3.2.0:
		\begin{verbatim}
			[ 1, 2, 3, 4 ]
			
			OR

			[
			   1,
			   2,
			   3,
			   4
			]
		\end{verbatim}
		\par
		Elements of a sequence \emph{must} be separated by commas.
		Also, please note that FHiCL sequences may not contain name-value pairs.
		\par
		FHiCL sequences may not contain comments.
		\par
		In order to access members of a FHiCL sequence, use of an \emph{hname}
		with the \emph{bracket index} is needed.
		\par
		Useage of the \emph{bracket index} requires that the sequence be associated with a name.
		\par
		For example, if we wanted to access the first element in seq1 from Example 3.2.0,
		and override the value to 5, it would look like this:
		\par
		Example 3.2.1:
		\begin{verbatim}
			seq1[0] : 5
		\end{verbatim}
		\par
		Similarly if we wanted to add a fifth value (let's say, the number 6?) to seq1,
		here's how it would look:
		\par
		Example 3.2.2:
		\begin{verbatim}
			seq1[4] : 6
		\end{verbatim}
		\par
		Please note that indices of FHiCL sequences begin at zero, not one.
\section{Include Statements}
	In order to import values from one FHiCL document into another,
	use of the FHiCL include statement is necessary.
	The behavior of this include statement is similar,
	and yet still different from the C/C++ include statement from which it was borrowed.
	\par
	The syntax for a FHiCL include statement begins with a pound or hash symbol (\#),
	followed by the unquoted string "include".
	The include is followed by \emph{exactly one white space}
	and then a quoted string containing the full name of the target FHiCL document.
	\par
	Example 4.1.0:
	\begin{verbatim}
		#include "filename.fcl"
	\end{verbatim}	
	\par
	Since the include statement shares a common starting character with that of one of the FHiCL comment notations,
	if the syntax of the include statement is incorrect,
	a FHiCL parser will treat the faulty include statement as a comment.
	\par
	If the file name within the quotes is invalid, however, the include statement will \emph{not} be treated as a comment.
\section{Prologue}
	FHiCL documents may contain prologue sections which may contain declarations used to create
	a standard configuration for a FHiCL document.
	\par
	The beginning of a prologue section is denoted by:
	\emph{BEGINPROLOGUE}
	and the end is denoted by:
	\emph{ENDPROLOGUE}.
	\par
	Example 5.1.0:
	\begin{verbatim}
		BEGINPROLOGUE
		   x : 5
		   y : 6
		ENDPROLOGUE
	\end{verbatim}
	All elements declared between the beginning and the end 
	do not exist within the parameter set unless specifically
	referenced within the document.
	\par
	Elements within a prologue section may be referenced,
	but may not be modified in any way otherwise.
	For instance, take the prologue section from Example 5.1.0.
	\par
	We can reference both \emph{x} and \emph{y} in the prologue section by using the reference notation:
	\par
	Example 5.1.1:
	\begin{verbatim}
		z : @local::x
		u : @local::y
	\end{verbatim}
	\par
	However we cannot modify x or y in any way.
\section{References}
	If at any point it becomes necessary to associate a name with a value of an existing name-value pair, or to 
	assign an existing value as a member of a sequence,
	the use of the FHiCL \emph{Reference} notation is needed.
	\par
	There are two versions of the reference notation in FHiCL, one being a local reference, 
	the other being a call or fetch to a predetermined database:
	\begin{itemize}
		\item @local::
		\item @db::
	\end{itemize}
	\par
	References may appear wherever a value normally is expected.
	The only prerequesite being that the name used within the reference notation must be the name of an existing element
	either within the current FHiCL document
	or within the called database.
	\par
	Example 6.1.0:
	\begin{verbatim}
		x : 5
		y : @local::x
		z : @db::t
	\end{verbatim}
	\par
	In this example, the value of \emph{y} is a reference to the pre-existing name-value pair \emph{x : 5}.
	The element named \emph{z}'s value is a reference to a name-value pair with name \emph{t} 
	which is to be found in a database.
	\par
	The names used within a FHiCL reference notation must be extremely specific in order to reference the correct element.
	\par
	Example 6.1.1:
	\begin{verbatim}
		x : 5
		x : 6
		tab1 :
		{
		   x : 1
		   y : 2
		   z : 3
		}
		y : @local::x
	\end{verbatim}
	\par
	In Example 6.1.1, we start with a definition \emph{ x : 5 },
	and then we override the value of \emph{x} with the number 6.
	The value of \emph{y} in this case will be a reference to the
	most recently encountered instance of \emph{x} within the same scope
	as \emph{y}.
	\par
	Essentially this means that the reference to \emph{x} here will have the value
	of \emph{ x : 6 },  
\section{Documents}
	FHiCL documents closely resemble FHiCL tables in that they are, for the most part,
	containers of name-value pairs.
	Unlike tables, however, FHiCL documents do not have braces around their elements
	and documents may contain prologue sections.
	\par
	A valid FHiCL document might look something like this:
	\par
	Example 7.1.0:
	\begin{verbatim}
		BEGINPROLOGUE
		   x : 5
		   y : 6
		ENDPROLOGUE
		
		tab1 :
		{
		   a : 1
		   b : 2
		   c : 3
   		}
		z : @local::x
		y : @local::y
	\end{verbatim}
\end{document}
