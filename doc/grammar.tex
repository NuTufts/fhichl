\documentclass[twoside]{memarticle}
\usepackage{marvosym} % to define \MVAt
\usepackage{rail}

\newcommand{\docnumber}%
  {draft 4}

\newcommand{\doctitle}%
  {DRAFT---{\color{magenta}Specification of the Fermilab Hierarchical Configuration Language}---DRAFT}

\newcommand{\shorttitle}%
 {DRAFT---Specification of the FHiCL---DRAFT}

\newcommand{\authors}%
  {
  Marc~Paterno \\
  Randolph~J.~Herber
  }

\hypersetup%
  { pdfauthor={}%
  , pdftitle={\doctitle}%
  , pdfkeywords={}%
  , pdfsubject={}%
  }

\tightlists

%
% Stuff to control the behavior of rail.
%
\railoptions{-hia}   % \railoptions{-h}
\railparam{\setlength{\leftmargin}{\leftmargini}}
\railnamefont{\rmfamily\itshape\color{magenta}}
%
\railalias{alphac}{a-zA-Z}
\railalias{at}{\MVAt}
\railalias{cbrace}{\}}
\railalias{cquote}{'}
\railalias{cr}{\char"5C\char"5C}
\railalias{digitc}{0-9}
\railalias{dollar}{\$}
\railalias{dquote}{"}
\railalias{greater}{\textgreater}
\railalias{hat}{\textasciicircum}
\railalias{less}{\textless}
%\railalias{nonzero}{1-9}
\railalias{obrace}{\{}
\railalias{oquote}{`}
\railalias{printc}{printable}
\railalias{rsolidus}{\textbackslash}
\railalias{space}{\textvisiblespace}
\railalias{star}{\textasteriskcentered}
\railalias{tilde}{$\sim$}
\railalias{uline}{\_}
\railalias{vbar}{\textbar}
%
\railterm{alphac,at,cbrace,cquote,cr,dollar,dquote,greater,hat,less,obrace,oquote,rsolidus,space,star,tilde,uline,urifragment,vbar}

\makeindex

\begin{document}
\topmatter
\chapter{Introduction}

\section{Purpose of this document}
This document provides the formal specification
for the \emph{Fermilab Hierarchical Configuration Language},
FHiCL.
It uses extended Backus-Naur format (EBNF) to describe the language grammar.
The EBNF is presented as ``railroad'' diagrams.
For purposes of pronunciation, the acronym rhythms with ``fickle."

\section{Notation used in this document}

Any \emph{nonterminal} is formatted in a square box, like:
\begin{rail}
  nonterminal;
\end{rail}

Any \emph{terminal} text is formatted in a box with rounded corners, like:
\begin{rail}
  'terminal';
\end{rail}

%A list of all terminals of FCL is provided in Appendix~\ref{app:terminals}

Arrows are used to show the direction in which diagrams should be read.

\chapter{Preprocessing}

Conceptually, preprocessing is a separate processing step which occurs
before the tokenization step and which may be and probably will be
implemented in a separate program.

\begin{rail}
preprocessorlines : '\#' (whitespace ?) 'include' whitespace
                    \\ dquote 'urifragment' dquote
                    \\ (whitespace ?);
\end{rail}

Preprocessor lines are syntactically hash comments in the FHiCL.
When discovered, the processor line will be copied to the output
and then followed by the contents of the urifragment data entity.
Inclusions may be nested to a minimum depth of 10.
If there exists a FHiCL shell environment variable,
which is a comma separated list of URI headers and file name paths,
then a URL will be formed with each URI header from left to right
concatenated with a `/' (if the fragment does not start already
with a `/' and the urifragment.
The following URI schemes may be supported:
\texttt{file://[host]/path}, \\texttt{ftp:/},
\\texttt{http://} and \\texttt{https://}.
The first successful URL will be used.
If no URL is successful, then the preprocessor line produces no output
and no error indications.
If there does not exist a FHiCL shell environment variable,
then the urifragment will be interpreted as a file name
relative to the current working directory
if it does not begin with a `/' and
as an absolute file name if it does begin with a `/'.

\chapter{Configuration language syntax}

\section{Encodings}

The input will be encoded in ASCII, a seven bit code.
In eight-bit bytes, the high-order bit will be zero.
In strings, all 256 eight-bit codes may be presented using ASCII characters.
A pretty printer program may represent non-printable
bytes by octal or hexadecimal escape sequences.
The using application is responsible for recognizing
the format of such texts as binary data or wide
character texts.

\section{Tokenization}

Conceptually, tokenization is done by selecting the token type name and
string as the {\em first occurring longest match} in the following list
of patterns.
All patterns are anchored at the beginning of the remaining text.
All {\em whitespace}, {\em hashcomments} and {\em doublesoliduscomments}
are discarded as soon as they are recognized.
The token type name and string are placed in a tuple.
A vector of tuples is formed.

The FHiCL is whitespace delimited and is not line oriented,
except for the hascomments and doublesoliduscomments.

There are seven keyword tokens:
\texttt{true}, \texttt{false}, \texttt{infinity}, \texttt{PROLOG}, \texttt{END},
\texttt{nil} and \texttt{null}.

The keywords \texttt{nil} and \texttt{null} will be mapped internally
to a single appropriate value for the implementation language.

N.B., leading and trailing zeroes on numeric values are considered
significant and are preserved by the language.

If an error token is recognized, then the input is invalid.

A list of sub patterns to simplify the main list of
type names and patterns:

\begin{rail} 
octaldigit : ( '0' | '1' | '2' | '3' | '4' | '5' | '6' | '7' );
\end{rail}

\begin{rail} 
nonzerodigit : ( '1' | '2' | '3' | '4' | '5' | '6' | '7' | '8' | '9' );
\end{rail}

\begin{rail} 
digit : ( '0' | 'nonzerodigit' );
\end{rail}

\begin{rail} 
hexdigit : ( digit | 'a' | 'b' | 'c' | 'd' | 'e' | 'f'
                   | 'A' | 'B' | 'C' | 'D' | 'E' | 'F');
\end{rail}

\begin{rail} 
sign : ( '-' | '+' );
\end{rail}

\begin{rail} 
optionalsign : ( sign? );
\end{rail}

\begin{rail} 
exponent : ( ( 'E' | 'e' ) optionalsign
             ( digit | digit digit | digit digit digit ) );
\end{rail}

\begin{rail}
alpha : ( 'A through Z, a through z or \_' );
\end{rail}

The list of type names and patterns:

\begin{rail}
whitespace : (( space | 'tab' | 'newline' | 'newpage' )?);
\end{rail}

\begin{rail}
hashcomments : ( '\#' (('any-character, except newline')*) );
\end{rail}

\begin{rail}
doublesoliduscomments : ( '//' (('any-character, except newline')*) );
\end{rail}

\begin{rail}
true : ( 'true' );
\end{rail}

\begin{rail}
false : ( 'false' );
\end{rail}

\begin{rail}
nil : ( 'nil' );
\end{rail}

\begin{rail}
nil : ( 'null' );
\end{rail}

\begin{rail}
number : ( optionalsign 'infinity' );
\end{rail}

\begin{rail}
number : ( optionalsign ( digit +) (('.' (digit *))?) (exponent ?) );
\end{rail}

\begin{rail}
number : ( optionalsign '.' (digit +) (exponent ?) );
\end{rail}

\begin{rail}
number : ( sign (digit +) );
\end{rail}

\begin{rail}
ordinal : ( (digit +) );
\end{rail}

\begin{rail}
name: (alpha | uline) ((alpha|digit| uline) *);
\end{rail}

\begin{rail}
string : (( '"' (('any character except a " or reverse solidus' |
             rsolidus ('"' | rsolidus | 'r' | 'n' | 't' | 'v' |
                       'f' | 'b' | '0' | (octaldigit octaldigit octaldigit) |
                       (('x' | 'X') hexdigit hexdigit)) )*)
           '"')+);
\end{rail}

\begin{rail}
dot : '.';
\end{rail}

\begin{rail}
atsign: at;
\end{rail}

\begin{rail}
leftbrace: obrace;
\end{rail}

\begin{rail}
rightbrace: cbrace;
\end{rail}

\begin{rail}
leftbracket: '[';
\end{rail}

\begin{rail}
rightbracket: ']';
\end{rail}

\begin{rail}
comma: ',';
\end{rail}

\begin{rail}
is: ':';
\end{rail}

\begin{rail}
specifications : ((definition | override)*)
\end{rail}

\begin{rail}
error: 'any character';
\end{rail}

\section{High-level entities}

Whitespace, hashcomments and doublesolidus comments are dropped
when recognized;
therefore, they are not significant.
They will not be mentioned in the following grammar rules.

A sequence of strings will be concatenated.

\texttt{PROLOG} keyword and \texttt{PROLOG} \texttt{END}
keyword sequences, if present, must occur in complete sets.
All definitions occuring within such a set are omitted from
the output.

The \emph{document} is the highest-level construct
in FCL.
Any implementation of an FCL parser
processes a \emph{document}
as a single string.

An empty output is acceptable.

Null is an appropriate distinct value of the parser implementation language.

The output of a successful parse of a document is a
table of definitions.
Overrides are not included in the output table.
An override which does not find its search target is an error.
If multiple copies of a name occur,
then a later occurance overrides an earlier occurance.
An implementation may issue warning messages detailing such overrides.
A user may choose choose to treat such warnings as errors.


\begin{rail}
document : ((('PROLOG' specifications 'PROLOG' 'END') | specifications)*)
\end{rail}

\begin{rail}
	definition: name is value;
\end{rail}

\begin{rail}
	numeric: number | integer | ordinal;
\end{rail}

\begin{rail}
	value: atom | numeric | ( '(' numeric ',' numeric ')' ) |
               table | sequence | reference;
\end{rail}

\begin{rail}
	sequence: leftbracket (( value ((',' value)*) )?) rightbracket;
\end{rail}

\begin{rail}
	table: leftbrace (( definition ((',' definition)*) )?) rightbrace;
\end{rail}

\begin{rail}
        hname: name ('(' ('last' | ordinal) ')')?
        \\ ((('.' name|'[' integer ']')+)?);
\end{rail}

\begin{rail}
        reference: hname atsign ('db' | 'local');
\end{rail}

\begin{rail}
        override: hname is (value | reference);
\end{rail}

A {\em reference} is replaced by the value referenced.
It is an error if the value referenced does not exist.

An element of type \emph{override} is used to
change the value of an existing element in the vector of tuples
already built.
The tuple being modified must be named.
The version number indexes into a subvector of tuples with that name.
If the overrride extends a sequence and there are missing intermediate values,
then the intermediate values are set {\em .nil}.

In an {\em hname},
the optional field immediately after the inital {\em name} field supplies
a version value.
If omitted, then there must be only referencable value by that name and
that is the referenced value.
``(last)" means the highest available version.

\chapter{Configuration language semantics}
\section{High-level result of a successful parse}

The output of a successful parse of a document is a
vector of name and value tuples.
The user of FCL has a choice of which values in the vector are significant.

When the vector or a value is {\em pretty printed},
if an atom meets the pattern of a numeric,
then it is printed without quotation, regardless of its original quotation.
In the vector of tuples, all atoms are strings.
Complex numbers are formed into strings as well.

The implementation must process ``@local" references itself.
The implementation may process ``@db" references as it chooses,
including considering them to be errors.

\section{Representation of atoms}
In the parse results,
all \emph{atom}s
except for \texttt{null} and \emph{reference}s
are represented as character strings.
The atom \texttt{null} is represented by a 
value specified by the binding for a given programming language.
The resolution of \emph{reference}s is described in
section~\ref{sec:refs} below.

Each language binding
provides its own mechanism
for turning atoms of type \emph{integer}, \emph{real} and \emph{complex}
from their string representation
into the appropriate numerical representation.

\section{Resolution of \emph{ref}s\label{sec:refs}}
Atoms of type \emph{ref} are replaced
by the value indicated by the \emph{hname} part of the \emph{ref},
where the environment in which the \emph{hname} is evaluated is determined
by the \texttt{db} or \texttt{here} at the end of the \emph{ref}.

The presence of \texttt{here} indicates 
that the scope in which the \emph{hname} is to be sought
is the previously-read \emph{document} text.
The presence of \texttt{db} indicates
that the scope in which the \emph{hname} is evaluated
is the single database
to which the parser has access.
If the parser has no access to a database,
and a \emph{ref} which ends in \texttt{db} is encountered,
a parse failure results.
If,
in the appropriate scope,
the \emph{hname} in a \emph{ref} does not resolve to any \emph{value},
a parse failure results.

\appendix

\chapter{Differences between JSON and FCL\label{app:differences}}
The language described is,
by intention,
similar but not identical to the Javascript Object Notation (JSON),
described at \url{http://www.json.org}.

Where JSON uses the name \emph{object},
we use the name \emph{table}.
Where JSON uses the name \emph{array}
we use the name \emph{sequence}.

The configuration language differs from JSON in several ways.
We draw special attention to the following.
\begin{enumerate}

\item JSON requires that the names of members in an object be strings,
which in JSON are always delimited by double quotes.
In the configuration language,
names of members of objects are not quoted,
and are subject to different constraints;
approximately stated,
names must be suitable as variable names
in commonly-used programming languages.
See the grammar specification below
for an exact description of the constraints.

\item JSON allows documents to contain any Unicode character.
The configuration language restricts documents 
to contain only printable ASCII characters.
This choice was made for the configuration language
because some of the languages for which we require bindings
do not have convenient support for Unicode.

\item JSON does not directly support complex numbers.
The configuration language has direct support 
for specification of values
that are complex numbers.

\item JSON recognizes \emph{number} as a primitive data type.
In the configuration language,
numbers and strings are united into a common type \emph{atom}.
This choice was made for the configuration language
because we need to support producing a printed representation
for every value
that is identical to the representation in the configuration document.
\end{enumerate}

\chapter{Processing notes for URI inclusion}

\begin{enumerate}
\item Use recursion.
\item Arguments are:
\begin{enumerate}
\item File-like object to receive the output.
\item URL to be processed.
\item Stack of currently in-process URLs.
\end{enumerate}
\item If URL to be processed already is in stack,
      then return (possibly with error indication).
\item Add URL to stack top.
\item Open URL, using search procedure described in {\em Preprocessing}.
\item If URL open failed,
      then return (possibly with error indication).
\item Set current position to zero.
\item Processing loop:
\begin{enumerate}
\item Copy lines, adding line length to position.
\item When a include line is discovered:
\begin{enumerate}
\item Close URL reading.
\item Call this routine with:
\begin{enumerate}
\item File-like object to receive the output.
\item The just discovered URL to be processed.
\item Stack of currently in-process URLs.
\end{enumerate}
\item Handle or ignore any error
\item Reopen current URL and read to current position.
\end{enumerate}
\end{enumerate}
\item Pop current URL from stack.
\item Return normally to caller.
\end{enumerate}

%\chapter{Terminals\label{app:terminals}}
%\begin{fixme}
%	Would it be useful to have a table of all FCL terminals here?
%	If so, what information should be in the table?	
%\end{fixme}

\printindex

\end{document}
